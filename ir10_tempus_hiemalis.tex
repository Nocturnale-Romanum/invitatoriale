% !TEX TS-program = lualatex
% !TEX encoding = UTF-8

\documentclass[nocturnale-romanum.tex]{subfiles}

\ifcsname preamble@file\endcsname
  \setcounter{page}{\getpagerefnumber{M-nr10_tempus_adventus}}
\fi

\begin{document}
\feast{PT}{Proprium de Tempore}{Proprium de Tempore}{Proprium de Tempore}{1}{}{}{}{}{}{}
\addcontentsline{toc}{chapter}{Proprium de Tempore}

\feast{A1F1}{Dominicæ I et II Adventus}
	{Proprium de Tempore}{Tempus Adventus}{2}{}
	{I. classis - Semiduplex}{I. classis}{Adventus!Hebdomada 1 et 2}
	{}
	{}
\gscore{A1F1I}{I}{}{Regem venturum Dominum!In Dominicis}

\feast{A1F2}{In Feriis Hebdomadæ I et II Adventus}
	{Proprium de Tempore}{Tempus Adventus}{2}{}
	{I. classis - Semiduplex}{I. classis}{}
	{}
	{}
\gscore{A1F2I}{I}{}{Regem venturum Dominum!In Feriis}

\feast{A3F1}{Dominicæ III et IV Adventus}
	{Proprium de Tempore}{Tempus Adventus}{2}{}
	{I. classis - Semiduplex}{I. classis}{Adventus!Hebdomada 1 et 2}
	{}
	{}
\gscore{A3F1I}{I}{}{Prope est jam Dominus!In Dominicis}

\feast{A3F2}{In Feriis Hebdomadæ III et IV Adventus}
	{Proprium de Tempore}{Hebdomada I Adventus}{2}{}
	{I. classis - Semiduplex}{I. classis}{}
	{}
	{}
\gscore{A3F2I}{I}{}{Prope est jam Dominus!In Feriis}

\feast{1224}{In Vigilia Nativitatis Domini}
	{Proprium de Tempore}{Tempus Adventus}{2}{24 Decembris}
	{Duplex I. classis}{I. classis}{Jesu Christi, Domini nostri!Nativitas, vigilia}
	{}
	{}
\rubric{Sequens Invitatorium dicitur etiam, si Vigilia in Dominica venit.}
\gscore{1224I}{I}{}{Hodie scietis quia veniet}


\feast{1225}{In Nativitate\\Domini nostri Jesu Christi}
	{Proprium de Tempore}{Tempus Nativitatis}{1}{25 Decembris}
	{Duplex I. classis}{I. classis}{Jesu Christi, Domini nostri!Nativitas}
	{}
	{}
\gscore{1225I}{I}{}{Christus natus est}

\feast{1226}{In Festo S. Stephani Protomartyris}
	{Proprium de Tempore}{Tempus Nativitatis}{2}{26 Decembris}
	{Duplex II. classis}{II. classis}{Stephani}
	{cum Octava simplici. Omnia de Communi unius Martyris, pag.\ \pageref{M-UMEX}, præter ea quæ hic habentur propria.}
	{}
\gscore{1226I}{I}{}{Christum natum qui beatum}

\feast{1227}{In Festo S. Joannis\\Apostoli et Evangelistæ}
	{Proprium de Tempore}{Tempus Nativitatis}{2}{27 Decembris}
	{Duplex II. classis}{II. classis}{Joannis Evangelistæ}
	{cum Octava simplici. Omnia de Communi Apostolorum, pag.\ \pageref{M-APEX}, præter ea quæ hic habentur propria.}
	{}

\feast{1228}{In Festo Ss. Innocentium Martyrum}
	{Proprium de Tempore}{Tempus Nativitatis}{2}{28 Decembris}
	{Duplex II. classis}{II. classis}{Innocentium}
	{cum Octava simplici. Omnia de plurimorum Martyrum, pag.\ \pageref{M-PMEX}, præter ea quæ hic habentur propria.}
	{}

\feast{N1F1}{Dominica infra Octavam Nativitatis}
	{Proprium de Tempore}{Tempus Nativitatis}{2}{}
	{Semiduplex Dominica minor}{II. classis}{Jesu Christi, Domini nostri!Nativitas, dominica infra octavam}
	{Omnia de Nativitate Domini ut pag.\pageref{M-1225}, præter ea quæ hic habentur propria.}
	{}

\feast{1229}{S. Thomæ Cantuariensis Episcopi et Martyris}
	{Proprium de Tempore}{Tempus Nativitatis}{2}{29 Decembris}
	{Duplex}{(Commemoratio tantum)}{Thomæ Cantuariensis}
	{Antiphonæ et Psalmi de Feria. Invitatorium, Hymnus, Versus Nocturnorum et Responsoria II. et III. Nocturnorum ut in Communi unius Martyris pag.\ \pageref{M-UMEX}. Responsoria I. Nocturni ut infra.}
	{Antiphonæ et Psalmi de Nativitate Domini ut pag.\pageref{M-1225}, cum Responsoriis 1. et 2. ut infra.}

\feast{1230}{Die VI infra Octavam Nativitatis}
	{Proprium de Tempore}{Tempus Nativitatis}{3}{30 Decembris}
	{Feria}{II. classis}
	{}
	{}
\rubric{Omnia dicuntur ut in Dominica infra Octavam Nativitatis.}
\tedeumrubric

\feast{1231}{S. Silvestri Papæ et Confessoris}
	{Proprium de Tempore}{Tempus Nativitatis}{2}{31 Decembris}
	{Duplex}{(Commemoratio tantum)}{Silvestri Papæ}
	{Antiphonæ et Psalmi de Feria. Invitatorium, Hymnus, Versus Nocturnorum et Responsoria II. et III. Nocturnorum ut in Communi Confessoris Pontificis pag.\ \pageref{M-COPO}. Responsoria I. Nocturni ut in Dominica.}
	{}

\feast{0101}{In Circumcisione Domini\\et Octava Nativitatis}
	{Proprium de Tempore}{Tempus Nativitatis}{2}{1 Januarii}
	{Duplex II. classis}{II. classis}{Jesu Christi, Domini nostri!Circumcisio}
	{}
	{}
\rubric{Invitatorium \scorename{1225I}, et Hymnus \scorename{1225H} ut in Nativitate Domini, pag.\ \pageref{M-1225}.}

\feast{N2}{In Festo Sanctissimi Nominis Jesu}
	{Proprium de Tempore}{Tempus Nativitatis}{2}{Dominica inter Circumcisionem et Epiphaniam}
	{Duplex II. classis}{II. classis}{Jesu Christi, Domini nostri!Nomen}
	{}
	{}
\gscore{N2I}{I}{}{Admirabile nomen Jesu}

\feast{0102}{In Octava S. Stephani Protomartyris}
	{Proprium de Tempore}{Tempus Nativitatis}{2}{2 Januarii}
	{Simplex}{(Omittitur)}{Stephani!Octava}
	{}
	{}
\rubric{Antiphonæ, Psalmi et Versus Nocturni de Feria occurenti, reliqua ut in Festo.

\feast{0103}{In Octava S. Joannis\\Apostoli et Evangelistæ}
	{Proprium de Tempore}{Tempus Nativitatis}{2}{3 Januarii}
	{Simplex}{(Omittitur)}{Joannis Evangelistæ!Octava}
	{}
	{}
\rubric{Antiphonæ, Psalmi et Versus Nocturni de Feria occurenti, reliqua ut in Festo.

\feast{0104}{In Octava Ss. Innocentium Martyrum}
	{Proprium de Tempore}{Tempus Nativitatis}{2}{4 Januarii}
	{Simplex}{(Omittitur)}{Innocentium!Octava}
	{}
	{}
\rubric{Antiphonæ, Psalmi et Versus Nocturni de Feria occurenti, reliqua ut in Festo.

\feast{0105}{In Vigilia Epiphaniæ}
	{Proprium de Tempore}{Tempus Nativitatis}{2}{5 Januarii}
	{Semiduplex II. classis}{(Omittitur)}{Jesu Christi, Domini nostri!Epiphania, vigilia}
	{}
	{}
\rubric{Officium hujus Vigiliæ locum tenet Officii Dominicæ quæ occurrit a die 1 ad 5 Januarii, vel a superveniente Festo aut Dominica infra Octavam Epiphaniæ impeditur; proindeque gaudet omnibus privilegiis Dominicæ. Omnia dicuntur, ut in Octava Nativitatis.}

\feast{0106}{In Epiphania\\Domini nostri Jesu Christi}
	{Proprium de Tempore}{Tempus Nativitatis}{1}{6 Januarii}
	{Duplex I. classis}{I. classis}{Jesu Christi, Domini nostri!Epiphania}
	{}
	{}
\rubric{Invitatorium omittitur, juxta rubricas.}



\feast{E1F1}{In Festo sanctæ Familiæ\\Jesu, Mariæ, Joseph}
	{Proprium de Tempore}{Tempus Nativitatis}{2}{Dominica infra Octavam Epiphaniæ}
	{Duplex majus}{II. classis}{Jesu Christi, Domini nostri!Familia Sancta}
	{Quando die Octava Epiphaniæ in Dominicam inciderit, Sabbato præcedenti fit Officium de Sancta Familia. Sicubi tamen hoc Sabbato occurrat Festum Duplex \Rnum{1} classis, Officium de Sancta Familía cum Commemoratione ipsius Dominicæ anticipatur in proximiorem Feriam infra Octavam.}
	{}
\gscore{E1I}{I}{}{Christum Dei Filium Mariae}

\feast{E0B}{Infra Octavam et in Octava Epiphaniæ}
	{Proprium de Tempore}{Tempus Nativitatis}{2}{7 usque ad 13 Januarii}
	{Semiduplex}{IV. classis}{}
	{}
	{}
\gscore{0106I}{I}{}{Christus apparuit nobis}

\rubric{In Feriis post Octavam Epiphaniæ, Invitatorium per Annum, ut infra.}

\feast{F1w}{In Dominicis post Epiphaniam}{Proprium de Tempore}{Tempus post Epiphaniam}{2}{}{}{}{}{}{}

\gscore{F1Iw}{I}{}{Adoremus Dominum}

\feast{F2w}{Feria II}{Proprium de Tempore}{Tempus post Epiphaniam}{2}{}{}{}{}{}{}
\gscore{F2I}{I}{}{Venite exsultemus}
\rubric{In primo Psalmi versu non repetitur \normaltext{Veníte, exsultémus Dómino},
sed post repititum Invitatorium, statim subjungitur \normaltext{Jubilémus Deo}, etc.}

\feast{F3w}{Feria III}{Proprium de Tempore}{Tempus post Epiphaniam}{2}{}{}{}{}{}{}
\gscore{F3I}{I}{}{Jubilemus Deo}

\feast{F4w}{Feria IV}{Proprium de Tempore}{Tempus post Epiphaniam}{2}{}{}{}{}{}{}
\gscore{F4I}{I}{}{Deum magnum Dominum}

\feast{F5w}{Feria V}{Proprium de Tempore}{Tempus post Epiphaniam}{2}{}{}{}{}{}{}
\gscore{F5I}{I}{}{Regem magnum Dominum}

\feast{F6w}{Feria VI}{Proprium de Tempore}{Tempus post Epiphaniam}{2}{}{}{}{}{}{}
\gscore{F6I}{I}{}{Dominum Deum nostrum}

\feast{F7w}{Sabbato}{Proprium de Tempore}{Tempus post Epiphaniam}{2}{}{}{}{}{}{}
\gscore{F7I}{I}{}{Populus Domini}

\feast{7GF1}{Dominicæ in Septuagesima, Sexagesima, Quinquagesima}
	{Proprium de Tempore}{Dominica in Septuagesima}{2}{}
	{II. classis - Semiduplex}{II. classis}{Tempus Septuagesimæ}
	{}
	{}
\gscore{7GF1I}{I}{}{Praeoccupemus faciem}
\rubric{In primo autem Psalmi versu omittitur: \normaltext{præoccupémus fáciem ejus in confessióne, et in psalmis jubilémus ei;} et repetitur a Choro Invitatorium.}

\feast{7GF2}{In Feriis Temporis Septuagesimæ}
	{Proprium de Tempore}{Tempus Septuagesimæ}{2}{}
	{Simplex}{IV. classis}{}
	{}
	{}
\rubric{Invitatorium ut supra in Feriis per Annum, pag.\ \pageref{M-F2w}.}

\rubric{In Feria IV Cinerum et in Feriis post Cineres, Invitatorium ut supra in Feriis per Annum, pag.\ \pageref{M-F4I}.}


\feast{Q1F1}{In Dominicis et Feriis Quadragesimæ}
	{Proprium de Tempore}{Tempus Quadragesimæ}{2}{}
	{I. classis - Semiduplex}{I. classis}{Quadragesima}
	{}
	{}
\gscore{Q1I}{I}{}{Non sit vobis vanum}
\rubric{In Feriis autem tonus 7a uti potest.}

\feast{Q5F1}{In Dominicis et Feriis Tempore Passionis}
	{Proprium de Tempore}{Tempus Passionis}{2}{}
	{I. classis - Semiduplex}{I. classis}{Quadragesima!Tempus Passionis}
	{}
	{}
\gscore{Q5I}{I}{}{Hodie si vocem}

\rubric{In Feria V in Cœna Domini, in Feria VI in Parasceve, et in Sabbato Sancto, Invitatorium omittitur, et Matutinum absolute incipitur ab Antiphona primi Psalmi.}

\end{document}