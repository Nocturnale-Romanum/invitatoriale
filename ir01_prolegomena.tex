% !TEX TS-program = lualatex
% !TEX encoding = UTF-8

\documentclass[invitatoriale-romanum.tex]{subfiles}

\ifcsname preamble@file\endcsname
  \setcounter{page}{\getpagerefnumber{M-ir01_prolegomena}}
\fi

\begin{document}

\begin{titlepage}
\begin{center}
\null\vspace{5mm}
{\Large\sc{}Nocturnale Romanum}

\vspace{5mm}

{\large\sc{}Tomus I}

\vspace{3.5cm}

{\Huge{}INVITATORIALE}

\vspace{1cm}

{\Large\sc{}pro omnibus diebus\\per totum anni circulum}

\vspace{5mm}

{\large\sc{}secundum ordinem Divini Officii\\a Pio pp X restituti\\cum variationibus a Joanne pp XXIII instauratis}

\vspace{5mm}

{\large\sc{}cum Invitatoriis in cantu gregoriano\\et tonis Psalmi Invitatorii}

\vspace{3.5cm}

{\large\sc{}instrumentum laboris}

\vfill

MMXXIII

\end{center}
\end{titlepage}

\null

\feast{OR}{Prolegomena}
	{Prolegomena}{Prolegomena}{2}{}{}{}{}{}{}
\thispagestyle{empty}
\addcontentsline{toc}{section}{Prolegomena}

\begin{english}
	
\intermediatetitle{A taste of things to come}

{\setlength{\parindent}{5mm}\small

This small book is the first volume of what will, God willing, become a complete edition of the \emph{Nocturnale Romanum}. 
Its modest size and scope aim to give cantors a taste of things to come. 
It contains the Invitatories for the whole year of the traditional Roman Office (following the overhaul of ferial Invitatories by St. Pius X), 
and the relevant psalm tones.

Its melodies rely both on the most recent practical edition of chant books for the reformed Roman Office 
--- in our case, the 2019 \emph{Liber Hymnarius} --- and on critical comparison of dozens of ancient manuscripts.
Invitatories of more recent introduction have been given melodies of the existing tradition of Gregorian Chant.

This book attempts to enable the Church at large to render unto God a fitting praise in the present; 
this is in part why it uses rhythmic signs, based on the rules of the \emph{mora vocis}
and on the rhythmic indications of the adiastematic manuscripts, where they exist, 
because the faithful are accustomed to these signs more than to reproductions of ancient neumes: a table of equivalence is found below. 
This is also why this book is laid out so that it may be used by those celebrating Divine Office 
according to either the 1960 or \emph{Divino Afflatu} rubrics: 
the rubrics predating the reforms of the 1950s are found in the left column where relevant, 
and the applicable 1960 rubrics are found in the right column.

This book is also, and most importantly at present, a \textbf{draft}. 
We, the editing team, are entirely too few to bring it up to proper standards by ourselves. 
Please give your feedback on\\{\footnotesize\url{https://github.com/Nocturnale-Romanum/nocturnale-romanum/issues}}

}

\end{english}

\pagebreak
\null

\intermediatetitle{Un avant-goût de l'avenir}

{\setlength{\parindent}{5mm}\small

Ce livre est le premier volume de ce qui deviendra, si Dieu le veut, une édition complète du \emph{Nocturnale Romanum}.
Par son modeste contenu, il vise à donner aux chantres qui l'utiliseront un avant-goût de l'avenir. 
Il contient les invitatoires pour toute l'année liturgique selon l'office romain traditionnel 
(postérieur aux réformes de saint Pie X qui ont modifié les invitatoires fériaux), 
ainsi que les tons du psaume invitatoire.

Ses mélodies s'appuient à la fois sur l'édition pratique la plus récente des livres de chant pour l'office romain réformé 
--- en l'espèce, le \emph{Liber Hymnarius} de 2019 --- et sur l'étude critique de dizaines des plus anciens manuscrits.
Les invitatoires d'introduction récente ont reçu des mélodies tirées de la tradition grégorienne existante.

Ce livre veut permettre au plus grand nombre de fidèles de rendre à Dieu dès aujourd'hui un culte digne de lui; 
c'est pourquoi il emploie les signes rythmiques, 
selon les règles de la \emph{mora vocis} et d'après les indications rythmiques des manuscrits adiastématiques, 
puisque les fidèles sont plus habitués à ces signes qu'à la copie des neumes anciens:
une table de correspondance est donnée ci-après. 
C'est aussi la raison pour laquelle ce livre, autant que possible, 
permet la célébration de l'Office Divin selon les rubriques de 1960 ou celles
définies dans \emph{Divino Afflatu}: 
les rubriques antérieures aux réformes des années 1950 sont données en colonne de gauche, et celles du Code de 1960 en colonne de droite.

Ce livre est aussi, et surtout --- pour le moment --- un \textbf{brouillon}. 
L'équipe d'édition est beaucoup trop réduite pour en amener la qualité, par elle-même, à un niveau satisfaisant.
Merci de bien vouloir signaler les corrections nécessaires sur\\{\footnotesize\url{https://github.com/Nocturnale-Romanum/nocturnale-romanum/issues}}

}

\pagebreak

\null\vfill

\intermediatetitle{Tabella neumatum}

{\gresetnabc{1}{visible}
\gresetclef{invisible}
\gresetinitiallines{0}
\gregorioscore{\subfix{nocturnale-romanum/gabc/neumata}}
}

\vfill

\cleardoublepage


\end{document}