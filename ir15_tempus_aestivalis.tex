% !TEX TS-program = lualatex
% !TEX encoding = UTF-8

\documentclass[invitatoriale-romanum.tex]{subfiles}

\ifcsname preamble@file\endcsname
  \setcounter{page}{\getpagerefnumber{M-ir15_tempus_aestivalis}}
\fi

\begin{document}

\feast{P0F1}{Dominica Resurrectionis}
	{Proprium de Tempore}{Tempus Paschale ante Ascensionem}{1}{}
	{}{}{Jesu Christi, Domini nostri!Resurrectio}
	{}
	{}
\vspace{-4mm}
\feast{P0F1}{et in Dominicis et Feriis\\Tempore Paschali ante Ascensionem}
	{Proprium de Tempore}{Tempus Paschale ante Ascensionem}{2}{}
	{}{}{}
	{}
	{}
\gscore{P0I}{I}{}{Surrexit Dominus vere}

\feast{P5F5}{In Ascensione Domini}
	{Proprium de Tempore}{Tempus Paschale post Ascensionem}{1}{}
	{}{}{Jesu Christi, Domini nostri!Ascensio}
	{}
	{}
\vspace{-4mm}
\feast{P5F5}{et post Ascensionem}
	{Proprium de Tempore}{Tempus Paschale post Ascensionem}{2}{}
	{}{}{}
	{}
	{}
\gscore{P5F5I}{I}{}{Alleluia Christum Dominum}

\rubric{Præcedens Invitatorium dicitur usque ad Vigiliam Pentecostes inclusive.}

\feast{P7F1}{Dominica Pentecostes}
	{Proprium de Tempore}{Dominica Pentecostes}{1}{}
	{}{}{Pentecostes}
	{}
	{}
\gscore{P7I}{I}{}{Alleluia Spiritus Domini}

\rubric{Præcedens Invitatorium dicitur usque ad Sabbatum Quatuor Temporum Pentecostes inclusive.}

\feast{H1F1}{In Festo Sanctissimæ Trinitatis}
	{Proprium de Tempore}{In Festo Sanctissimæ Trinitatis}{2}{Dominica I post Pentecosten}
	{}{}{Trinitatis}
	{}
	{}
\gscore{H1I}{I}{}{Deum verum}

\rubric{In Feriis II, III et IV post Octavam Pentecostes, Invitatorium per Annum, ut infra pag.\ \pageref{M-F2s}.}

\feast{H1F5}{In Festo Sanctissimi Corporis Christi}
	{Proprium de Tempore}{In Festo Sanctissimi Corporis Christi}{2}{Feria V post Octavam Pentecosten}
	{Duplex I. classis}{I. classis}{Jesu Christi, Domini nostri!Corpus}
	{}{}
\gscore{H1F5I}{I}{}{Christum Regem adoremus}
\twocolrubric{Infra Octavam (et in Dominica infra Octavam) et in die Octava, Invitatorium ut in Festo.}{(Non habet Octavam.)}

\feast{H2F6}{In Festo Sacratissimi Cordis Jesu}
	{Proprium de Tempore}{In Festo Sacratissimi Cordis Jesu}{2}{Feria VI post Octavam Ss. Corporis Christi}
	{Duplex I. classis}{I. classis}{Jesu Christi, Domini nostri!Cor}
	{}{}
\gscore{H2F6I}{I}{}{Cor Jesu amore}
\twocolrubric{Infra Octavam (et in Dominica infra Octavam) et in die Octava, Invitatorium ut in Festo.}{(Non habet Octavam.)}

\feast{F1s}{In Dominicis post Pentecosten}
	{Proprium de Tempore}{Post Pentecosten}{2}{}
	{}{}{}
	{}
	{}
\rubric{Usque ad mensem Septembris inclusive:}
\gscore{F1Is}{I}{}{Dominum qui fecit nos}
\vspace{\baselineskip}
\rubric{In mensibus Octobris et Novembris:}
\gscore{F1Iw}{I}{}{Adoremus Dominum}

\feast{F2s}{Feria II}{Proprium de Tempore}{Post Pentecosten}{2}{}{}{}{}{}{}
\gscore{F2I}{I}{}{Venite exsultemus}

\pagebreak

\feast{F3s}{Feria III}{Proprium de Tempore}{Post Pentecosten}{2}{}{}{}{}{}{}
\gscore{F3I}{I}{}{Jubilemus Deo}

\feast{F4s}{Feria IV}{Proprium de Tempore}{Post Pentecosten}{2}{}{}{}{}{}{}
\gscore{F4I}{I}{}{Deum magnum Dominum\newline\null}

\feast{F5s}{Feria V}{Proprium de Tempore}{Post Pentecosten}{2}{}{}{}{}{}{}
\gscore{F5I}{I}{}{Regem magnum Dominum\newline\null}

\feast{F6s}{Feria VI}{Proprium de Tempore}{Post Pentecosten}{2}{}{}{}{}{}{}
\gscore{F6I}{I}{}{Dominum Deum nostrum\newline\null}

\feast{F7s}{Sabbato}{Proprium de Tempore}{Post Pentecosten}{2}{}{}{}{}{}{}
\gscore{F7I}{I}{}{Populus Domini}


\end{document}