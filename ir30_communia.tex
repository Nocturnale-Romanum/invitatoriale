% !TEX TS-program = lualatex
% !TEX encoding = UTF-8

\documentclass[invitatoriale-romanum.tex]{subfiles}

\ifcsname preamble@file\endcsname
  \setcounter{page}{\getpagerefnumber{M-ir30_communia}}
\fi

\begin{document}
\feast{CS}{Commune Sanctorum}{Commune Sanctorum}{Commune Sanctorum}{1}{}{}{}{}{}{}
\addcontentsline{toc}{chapter}{Commune Sanctorum}

\feast{VIAP}{In Vigiliis Apostolorum}
	{Commune Sanctorum}{In Vigiliis Apostolorum}{2}{}
	{}{}{}
	{}{}
\invitferia

\feast{APEX}{Commune Apostolorum\\extra Tempus Paschale}
	{Commune Sanctorum}{Commune Apostolorum extra T. P.}{2}{}
	{}{}{}
	{}
	{}
\addcontentsline{toc}{section}{Commune Apostolorum et Evangelistarum}
\rubricdoublefeasts
\gscore{APEXIb}{I}{}{Regem Apostolorum (tonus festivus)}
\rubricsolemnfeasts
\gscore{APEXIa}{I}{}{Regem Apostolorum (tonus solemnis)}
\rubrictp
\gscore{APTPI}{I}{}{Regem Apostolorum (tempore paschali)}

\pagebreak

\feast{UMEX}{Commune unius aut plurimorum Martyrum}
	{Commune Sanctorum}{Commune Martyrum}{2}{}{}{}{}{}{}
\addcontentsline{toc}{section}{Commune Martyrum}

\rubricsimplefeasts
\gscore{UMEXIa}{I}{}{Regem Martyrum (tonus simplex)}
\rubricdoublefeasts
\gscore{UMEXIb}{I}{}{Regem Martyrum (tonus festivus)}
\rubricsolemnfeasts
\gscore{UMEXIc}{I}{}{Regem Martyrum (tonus solemnis)}
\rubrictp
\gscore{MRTPI}{I}{}{Exsultent in Domino}


\feast{COPO}{Commune Confessoris\\Pontificis aut non Pontificis,\\et Doctorum Ecclesiæ, atque Abbatum}
	{Commune Sanctorum}{Commune Confessoris}{2}{}{}{}{}{}{}
\addcontentsline{toc}{section}{Commune Confessoris}

\rubricsimplefeasts
\gscore{COPOIa}{I}{}{Regem Confessorum (tonus simplex)}
\rubricdoublefeasts
\gscore{COPOIb}{I}{}{Regem Confessorum (tonus festivus)}
\rubricsolemnfeasts
\gscore{COPOIc}{I}{}{Regem Confessorum (tonus solemnis)}
\pagebreak
\rubrictp
\gscore{COPOId}{I}{}{Regem Confessorum (tempore paschali)}



\feast{MU}{Commune Virginum}
	{Commune Sanctorum}{Commune Virginum}{2}{}{}{}{}{}{}
\addcontentsline{toc}{section}{Commune Virginum}

\rubricsimplefeasts
\gscore{MUVXIa}{I}{}{Regem Virginum (tonus simplex)}
\rubricdoublefeasts
\gscore{MUVXIb}{I}{}{Regem Virginum (tonus festivus)}
\pagebreak
\rubricsolemnfeasts
\gscore{MUVXIc}{I}{}{Regem Virginum (tonus solemnis)}
\rubrictp
\gscore{MUVXId}{I}{}{Regem Virginum (tempore paschali)}

\feast{MU}{Commune Mulierum}
	{Commune Sanctorum}{Commune Mulierum}{2}{}{}{}{}{}{}
\addcontentsline{toc}{section}{Commune Mulierum}
\rubric{Pro una non Virgine:}
\gscore{MUNXIa}{I}{}{Laudemus (unius non virginis)}
\pagebreak
\rubric{Pro plurimis non Virginis:}
\gscore{MUNXIb}{I}{}{Laudemus (plurimarum non virginis)}


\feast{CDED}{Commune Dedicationis Ecclesiæ}
	{Commune Sanctorum}{Commune Dedicationis Ecclesiæ}{2}{}{}{}{}{}{}
\addcontentsline{toc}{section}{Commune Dedicationis}
\gscore{CDEDI}{I}{}{Domum Dei}

\twocolrubric{Infra Octavam et in die Octava Invitatorium ut in Festo.}{(Non habet Octavam.)}

\pagebreak

\feast{CBMV}{Commune Beatæ Mariæ Virginis}
	{Commune Sanctorum}{Commune Beatæ Mariæ Virginis}{2}{}{}{}{}{}{}
\addcontentsline{toc}{section}{Commune Beatæ Mariæ Virginis}

\gscore{CBMVI}{I}{}{Sancta Maria}

\feast{CSMS}{De Sancta Maria in Sabbato}
	{Commune Sanctorum}{De Sancta Maria in Sabbato}{2}{}
	{}{}{Sabbato Sanctae Mariae}
	{}{}

\gscore{CSMSI}{I}{}{Ave Maria gratia plena}

\pagebreak

\feast{OPBM}{Officium Parvum\\Beatæ Mariæ Virginis}
	{Officium Parvum B. M. V.}{Officium Parvum B. M. V.}{2}{}{}{}{}{}{}

\rubric{Invitatorium \scorename{CSMSI}, ut supra. Tempore Paschali in fine Invitatorii \normaltext{Alleluia} non additur.}


\feast{ODEF}{Officium Defunctorum}
	{Officium Defunctorum}{Officium Defunctorum}{2}{}
	{}{}{Defunctorum!Officium}
	{}{}
\addcontentsline{toc}{section}{Officium Defunctorum}

\gscore{ODEFI}{I}{}{Regem cui omnia vivunt}

\end{document}